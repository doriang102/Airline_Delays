%%%%%%%%%%%%%%%%%%%%%%%%%%%%%%%%%%%%%%%%%%%%%%%%%%%%%%%%%%%%%%%%%%%%%%
% writeLaTeX Example: Academic Paper Template
%
% Source: http://www.writelatex.com
% 
% Feel free to distribute this example, but please keep the referral
% to writelatex.com
% 
%%%%%%%%%%%%%%%%%%%%%%%%%%%%%%%%%%%%%%%%%%%%%%%%%%%%%%%%%%%%%%%%%%%%%%
% How to use writeLaTeX: 
%
% You edit the source code here on the left, and the preview on the
% right shows you the result within a few seconds.
%
% Bookmark this page and share the URL with your co-authors. They can
% edit at the same time!
%
% You can upload figures, bibliographies, custom classes and
% styles using the files menu.
%
% If you're new to LaTeX, the wikibook is a great place to start:
% http://en.wikibooks.org/wiki/LaTeX
%
%%%%%%%%%%%%%%%%%%%%%%%%%%%%%%%%%%%%%%%%%%%%%%%%%%%%%%%%%%%%%%%%%%%%%%
\documentclass[twocolumn,showpacs,%
  nofootinbib,aps,superscriptaddress,%
  eqsecnum,prd,notitlepage,showkeys,10pt]{revtex4-1}

\usepackage{amssymb}
\usepackage{amsmath}
\usepackage{graphicx}
\usepackage{dcolumn}
\usepackage{hyperref}

\begin{document}

\title{Predicting Airline Delays: A Comparison of Models and Features}
\author{Dorian Goldman}
\affiliation{Conde Nast, Data Scientist,\\ Columbia University, Adjunct Professor of Data Science\\ }

\begin{abstract}
In this note, we compare different methodologies for predicting airline delays. We focus on all airline delays which are made publicly available on the website of
The Bureau of Transportation Statistics \cite{}. We compare different modeling methodologies including classification and regression before and after an enriched set of features
is introduced. The features which we add to provide supporting predictive accuracy are weather data from the NOAA, plane data from TODO and publicly available holiday data. We find
that all three of the above improve predictive performance for regression and classification. However by implementing an auto regressive time series model to predict the mean delay on days prior 
to the day being evaluated, we find the strongest increase in performance. A more logical  and robust approach would be to model the delays as a Poisson regression with the time dependent mean being a linear function of the features above, given the observed
distributions of the data. However given the time constraints, this approach was not investigated further.  This observation confirms the paradigm that carefully constructed features are often more predictive than more complex models in general.
\end{abstract}

\maketitle

\section{Introduction}

Airline delays are becoming an increasingly problematic issue for many people around the world. With increased competition amongst airlines \cite{} and increasingly restrictive choices for customers \cite{} along with limited resources
\cite{}, the problem has been further amplified. Flights which are delayed or canceled have cost consumers millions of dollars over the past few years alone \cite{}, and this trend is only seeming to increase \cite{}. To this date, there has been little effort to provide reasonably accurate predictions about delays which could potentialy save consumers a great deal of money and headache. While there have been modest attempts to understand this phenomenon \cite{}, most approaches fall short of a robust
method of modeling these delays that can be used by the consumer in advance. In this paper, we aim to approach the problem from the approach of machine learning and to compare the power of various models to that of careful feature selection. 

\section{Discussion of Main Results}

\subsection{Regression on Departure Delay}



Use \texttt{section}s and \texttt{subsection}s to organize your document. \LaTeX{} handles all the formatting and numbering automatically. Use \texttt{ref} and \texttt{label} for cross-references --- this is Section~\ref{sec:examples}, for example.

\subsection{Classification on Departure Delay}


\subsection{Most prevalent features}
Use \texttt{tabular} for basic tables --- see Table~\ref{tab:widgets}, for example. You can upload a figure (JPEG, PNG or PDF) using the files menu. To include it in your document, use the \texttt{includegraphics} command (see the comment below in the source code).

% Commands to include a figure:
%\begin{figure}
%\includegraphics[width=\textwidth]{your-figure's-file-name}
%\caption{\label{fig:your-figure}Caption goes here.}
%\end{figure}

\begin{table}
\centering
\begin{tabular}{l|r}
Item & Quantity \\\hline
Widgets & 42 \\
Gadgets & 13
\end{tabular}
\caption{\label{tab:widgets}An example table.}
\end{table}

\subsection{Conclusion}

\LaTeX{} is great at typesetting mathematics. Let $X_1, X_2, \ldots, X_n$ be a sequence of independent and identically distributed random variables with $\text{E}[X_i] = \mu$ and $\text{Var}[X_i] = \sigma^2 < \infty$, and let
$$S_n = \frac{X_1 + X_2 + \cdots + X_n}{n}
      = \frac{1}{n}\sum_{i}^{n} X_i$$
denote their mean. Then as $n$ approaches infinity, the random variables $\sqrt{n}(S_n - \mu)$ converge in distribution to a normal $\mathcal{N}(0, \sigma^2)$.

\subsection{Lists}

You can make lists with automatic numbering \dots

\begin{enumerate}
\item Like this,
\item and like this.
\end{enumerate}
\dots or bullet points \dots
\begin{itemize}
\item Like this,
\item and like this.
\end{itemize}

\begin{acknowledgments}

We thank\dots

\end{acknowledgments}

\end{document}